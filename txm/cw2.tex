\documentclass[12pt, letterpaper, titlepage]{article}
\usepackage[left=3.5cm, right=2.5cm, top=2.5cm, bottom=2.5cm]{geometry}
\usepackage[MeX]{polski}
\usepackage[utf8]{inputenc}
\usepackage{graphicx}
\usepackage{enumerate}
\usepackage{amsmath} %pakiet matematyczny
\usepackage{amssymb} %pakiet dodatkowych symboli
\title{Ćwiczenie 2}
\author{Paweł Nieczepa}
\date{Październik 2022}
\begin{document}
\maketitle

\begin{center}
\Huge 
\textbf{Przepis na Kopytka}
\end{center}

\Large
Poznaj moje najlepsze kopytka ziemniaczane, tradycyjne. To chyba najczęściej gotowane kluski i super pomysł na prosty obiad. Te super pyszne kopytka podaję zazwyczaj z sosem i mięsem lub też z bułką tartą.\newline

\noindent - stary, sprawdzony przepis na kopytka \newline
- dokładny opis i idealne proporcje składników \newline
- propozycja podania oraz wiele ciekawych porad \newline
\normalsize

\noindent\textbf {Czas przygotowania:} 40 minut \newline
\textbf {Liczba porcji:} 1230 g kopytek \newline

\noindent\textbf {Kaloryczność kcal:} 145 w 100 g kopytek \newline
\textbf {Dieta:} wegetariańska \newline

\section{Składniki:}
\subsection{1 kg ugotowanych ziemniaków}
\subsection{9 czubatych łyżek mąki pszennej tortowej}
\subsubsection{około 230 g}
\subsection{1 średnie lub małe jajko}
\subsection{płaska łyżeczka soli}

\section{Dodatkowe informacje:}

\begin{enumerate}
\item Szklanka ma u mnie pojemność 250 ml. 
\item Potrzebujesz również: miska, praska do ziemniaków lub maszynka do mielenia, drewniana stolnica, duży garnek, cedzak do łowienia kopytek.
\item Z podanej ilości składników powinno wyjść około 1230 gramów kopytek. Będą to cztery duże, lub sześć mniejszych porcji.
\item Kalorie policzone zostały na podstawie użytych przeze mnie składników. Jest to więc orientacyjna ilość kalorii liczona na podstawie produktów, których użyłam w danym przepisie. Podając kaloryczność nie uwzględniłam karkówki w sosie.

\end{enumerate}

\section{Przygotowanie:}
Aby zrobić kopytka potrzebujesz jeden kilogram ugotowanych już ziemniaków. Ugotowane ziemniaki najlepiej jest zmielić w maszynce lub przepuścić przez praskę jeszcze lekko ciepłe. Pamiętaj o tym, że ciepłe ziemniaki łatwiej jest przepuścić przez praskę niż zimne z lodówki, które robią się wtedy twarde. Jeśli jednak mielisz ziemniaki w maszynce, to możesz też użyć ziemniaków z obiadu, z dnia poprzedniego. Wówczas mogą być też zimne. 

Jeśli dopiero planujesz gotować ziemniaki, to polecam użyć około 1400 gramów ziemniaków (waga przed obraniem). Ziemniaki włóż do osolonej wody i gotuj do miękkości. Ważne jest, by zawsze dobrze osuszyć ziemniaki po ugotowaniu. Polecam po odlaniu wody postawić jeszcze na minutę garnek z samymi ziemniakami na małej mocy palnika, by odparować resztki wody z pomiędzy ziemniaków. Zmielone na gładko ziemniaki umieść w szerszej misce.

Porada: Na zdjęciu widać, że moje ziemniaki maiły optymalną ilość skrobi. Nie były ani zbyt sypkie, ani rzadkie i kleiste. Te bardzo kleiste ziemniaki z małą zawartością skrobi są po zmieleniu bardzo rzadkie, klejące i sprawiają wrażenie lekko przezroczystych. Takich ziemniaków unikamy, ponieważ trzeba do nich potem dodać znacznie więcej mąki, nawet do 100 gramów więcej. 
Na stolnicę przesiej dziewięć czubatych łyżek mąki pszennej. Będzie to około 230 gramów mąki. Wyłóż gładką masę ziemniaczaną. Na środek wbij jedno małe lub średnie jajko. Dodaj też płaską łyżeczkę soli. Masa ziemniaczana może być jeszcze lekko ciepła.

Całość wyrób na gładką masę bez żadnych grudek. Masa powinna być bardzo plastyczna. Kolor masy na kopytka zależy od tego, jakich ziemniaków użyjesz. Kopytka mogą zatem wyjść bardzo jasne lub też lekko żółte. Ja użyłam ziemniaków żółtych przeznaczonych do gotowania. 

Porada: Jak już wspomniałam o tym wcześniej, pamiętaj że ziemniaki są różne. Bywają rzadsze lub bardziej sypkie odmiany ziemniaków. Być może do masy będzie trzeba dodać odrobinę więcej mąki. Należy jednak pamiętać, by nie dać jej za dużo, bo kopytka będą zbyt twarde.
W dużym garnku można już zacząć gotować wodę. Posól ją płaską łyżką soli w momencie, gdy zacznie się gotować. Na obsypanej mąką stolnicy umieść 1/4 ciasta na kopytka i rozwałkuj dłońmi na sznurek grubości około 2 cm. Ucinaj nożem zgrabne kopytka o długości około 1,5-2 cm. 

Z podanej proporcji składników wychodzi mięciutka i delikatna masa, więc możesz śmiało podsypywać wałeczki mąką. 
Oprószone lekko mąką przenoś je na cedzak i zanurzaj w gotującej się wodzie. Od razu lekko i ostrożnie zamieszaj kopytka. Po wypłynięciu kopytek na wierzch odczekaj jeszcze około 90 sekund. Wyłów wszystkie kopytka, poczekaj aż odcedzisz nadmiar wody i umieść na szerokim talerzu lub salaterce. W ten sposób ugotuj wszystkie porcje kopytek.

\end{document}
